\subsection{Recommended Implementation}
The system will be composed by:
\begin{itemize}
    \item Client : it can be a mobile or a web application
    \item Web Server
    \item Application Server
    \item Database
    \item External APIs (such as OpenWeather and OpenStreetMap)
\end{itemize}

\subsection{Implementation Plan}
The above components should be developed following a bottom-up logic, therefore, avoiding difficulties in integration and testing.\bigskip

Firstly some basic functionalities should be completely developed so that the other related functionalities can then be added and tested.\newline
The first module that should be written is the one that provides authentication and account creation interfacing with the DBMS. 
Because of that, the system can soon reach partial functionality.\newline 
Then can be developed the web application using the REST API which includes all the main functionalities of the users.
The next steps to complete the system should be:
\begin{itemize}
    \item The client interfaces
    \item The coupling with external APIs
\end{itemize}

\subsection{Testing Plan}
Testing should be done throughout all stages of development, following a bottom up approach.\bigskip

Unit tests should be written starting from the basic component so that tests for other components that
use them can be written relying on the fact that the subcomponents are already tested. This speeds up
debugging since it’s easy to locate at which level of the component hierarchy there is a problem.

\subsection{Integration Plan}
In this section there is be a description about how the components are integrated and communicate.
The first component to build is the DBMS, followed by the main application components that exploit it.
After this, we have to integrate the API communication between the System and the external services that will be used.
t this point it is possible to integrate the User Manager, which permits to Users to exploit all the functionalities.
Finally, it is possible to integrate the web server module with the mobile application module and the browser with the web server module.
