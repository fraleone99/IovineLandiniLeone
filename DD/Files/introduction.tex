\subsection{Purpose}
This application aims to improve the agricultural process of Telangana by allowing farmers to communicate with each other and with agronomists.
Furthermore, a comprehensive view of problems and results is provided to policymakers.
Farmers and  Agronomists should be able to access the application's functionalities through a mobile app. Policymakers should be able to access the application's functionalities through a web app.
\subsection{Scope}
The application will enable different actors to use the application:

\begin{itemize}
    \item The \textbf{farmer} will insert data about his production and information about his problems. He can check the weather and personalized suggestion based on his types of crops and the weather of his zone. In addition, he can ask a question on the forum and can reply to questions of other farmers. In case of serious problems, he can send a private help request to the agronomist responsible for their zone.
    \item The \textbf{agronomist} can organize his visits to the farms, can check the weather and register good practices. In addition, he can answer help requests and forum questions, he can set threshold for production.
    \item The \textbf{policymaker} can visualize a dashboard with relevant information about farmers and the agricultural production of Telangana.
\end{itemize}

\subsection{Document Structure}
\begin{itemize}
    \item \textbf{Section 1: Introduction}\newline
    This section offers a brief description of the document that will be presented, with all the definitions, acronyms and abbreviations that
    will be found reading it.
    \item \textbf{Section 2: Architectural Design} \newline
    This section provides a more detailed description of the architecture of the system. The first part describes the chosen paradigm and 
    the overall split of the system into several layers. Then, a high-level description of the system is provided, 
    together with a presentation of the modules composing its nodes. Finally, there is a concrete description of the tiers forming the System.
    \item \textbf{Section 3: User Interface} \newline
    This section contains several mockups of the application, together with some charts useful to
    understand the correct flow of execution of it. The presented mockups refers to the client-side experience.
    \item \textbf{Section 4: Requirements Traceability} \newline
    This section acts as a bridge between the RASD and DD document, providing a complete mapping of the requirements described in the RASD 
    to the logical modules presented in this document. 
    \item \textbf{Section 5: Implementation, Integration and Test Plan} \newline
    The last section describes the procedures followed for implementing, testing and integrating the
    components of our System. There will be a detailed description of the core functionalities of it, together with a complete report about how to
    implement and test them.
\end{itemize}