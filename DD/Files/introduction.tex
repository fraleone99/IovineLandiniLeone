\subsection{Purpose}
This application aims to improve the agricultural process of Telangana by allowing farmers to communicate with each other and with agronomists.
Furthermore, a comprehensive view of problems and results is provided to policymakers.
Farmers and  Agronomists should be able to access the application's functionalities through a mobile app. Policymakers should be able to access the application's functionalities through a web app.
\subsection{Scope}
The application will enable different actors to use the application:

\begin{itemize}
    \item The \textbf{farmer} will insert data about his production and information about his problems. He can check the weather and personalized suggestion based on his types of crops and the weather of his zone. In addition, he can ask a question on the forum and can reply to questions of other farmers. In case of serious problems, he can send a private help request to the agronomist responsible for their zone.
    \item The \textbf{agronomist} can organize his visits to the farms, can check the weather and register good practices. In addition, he can answer help requests and forum questions, he can set threshold for production.
    \item The \textbf{policymaker} can visualize a dashboard with relevant information about farmers and the agricultural production of Telangana.
\end{itemize}