\subsection{Product perspective}
\subsubsection{Class diagrams}
This section will present the high-level UML class diagrams of the application.
\\\\
\begin{figure}[H]
    \centering
    \includegraphics[width=\textwidth,height=\textheight,keepaspectratio]{Images/HighLevelUMLcut.png}
    \caption{\label{fig:high_level_uml}High level class diagram}
\end{figure}

\bigskip
\paragraph{Additional notes on the class diagram}
\begin{itemize}
    \item The agronomist attribute dailyPlan is a collection of FarmInspection ordered by date and time
    \item Both farmers and agronomists can create a suggestion responding to a farmer request
\end{itemize}

\newpage
\subsubsection{State machine diagrams}
The following diagrams are meant to give a high-level description of the states' evolution during the system processes.
\\\\

\begin{figure}[H]
    \includegraphics[width=\textwidth,height=\textheight,keepaspectratio]{Images/farmerChecksWeather.png}
    \caption{Statechart of a farmer checking weather forecast}
    \label{fig:statechart_farmer_weather}
\end{figure}

\begin{figure}[H]
    \includegraphics[width=\textwidth,height=\textheight,keepaspectratio]{Images/farmerCreatesThread.png}
    \caption{Statechart of the lifetime of a discussion thread on the forum}
    \label{fig:statechart_farmer_thread}
\end{figure}

\begin{figure}[H]
    \includegraphics[width=\textwidth,height=\textheight,keepaspectratio]{Images/agronomistDailyPlan.png}
    \caption{Statechart of an agronomist managing his daily plan}
    \label{fig:statechart_agronomist_plan}
\end{figure}

\bigskip
\begin{figure}[H]
    \includegraphics[width=\textwidth,height=\textheight,keepaspectratio]{Images/agronomistVisitFarm.png}
    \caption{Statechart of the lifetime of a farm visit}
    \label{fig:statechart_agronomist_visit}
\end{figure}

\newpage
\subsection{Product functions}
\emph{DREAM} offers Farmers the ability to ask for help from other farmers or an agronomist, 
they can check the weather forecast and see if there are suggestions for them. 
\emph{DREAM} create a leaderboard of the farmers that both agronomist and policymakers can visualize. 
The first ones can discover good practices and help farmers that are performing badly, 
policymakers instead can see the best farmer, reward them and see if the ones who accepted 
suggestions are performing better. 
In addition, \emph{DREAM} offers agronomists the possibility of visualizing, modifying or confirming their daily plan.

\paragraph{Common user functions} These functions are available to all Users:
\begin{itemize}
    \item \textbf{Registration and Login}\\
          Farmer should be able to create a personal account for the App using personal email and password. By logging in with their account, 
          Farmer can access the other functions.
\end{itemize}

\paragraph{Basic Farmer functions} These functions are available to all Farmer:
\begin{itemize}
    
    \item \textbf{Check weather forecast}\\
          \emph{DREAM} retrieves meteorological short-term and long-term data from an external source and makes them available 
          to farmers. They can check the forecasts on the app and if there are suggestions for their farm these are shown to them.
           These suggestions are created when an agronomist helps a farmer in a similar situation.
    \item \textbf{Request help}\\
          A farmer in case of a problem can ask for the direct help of an agronomist. He can use a section to send a help request,
           when this functionality is used a notification is sent to the agronomist responsible for the location of the farm with informations about the problem and the farm.
    \item \textbf{Create discussion on forum}\\
         \emph{DREAM} has a forum section on which farmers can create a discussion specifying the type of problem and eventual specificity of their farm. 
    \item \textbf{Reply to forum discussion}
        After access to the forum section of \emph{DREAM}, the farmer can select a recent, not closed discussion and post a reply.
    \item \textbf{Close a discussion}
        A farmer after receiving an effective reply can mark the answer as effective and close the discussion.
\end{itemize}