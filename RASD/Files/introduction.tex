\subsection{Purpose}
The covid-19 pandemic has greatly highlighted the need of building a resilient food system in India, 
where agriculture plays a pivotal role. 
This need is increased by the problems due to climate change that will impact everything from productivity to 
livelihoods across food and farm systems and is predicted to result in a 4\%-26\% loss in net farm income towards the end of the century.
In addition, according to Harvard Business Review, food demand is expected to increase between 59\% to 98\% by 2050.\\
For this reason policy makers, citizens, agronomists, and farmers should share data and information to achieve better results 
\\\\
Telangana region is an extended and populous state of India, whose economy is mainly driven by agriculture. 
To address the described above problem, Telangana's government want to design, 
develop and demonstrate anticipatory governance models for food system using digital 
public goods and community-centric approaches to strengthen data-driven policy-making in the state.
\\\\
The application aims to enable the acquisition, communication and combination of data provided by Telangana policymakers, 
farmers and agronomists as:
\begin{itemize}
    \item meteorological forecasts
    \item farmers' production
    \item amount of used water
    \item soil humidity
    \item agronomists' report
\end{itemize}
\bigskip
The product will allow policymakers to identify farmers who are performing well and those who are performing badly. 
As the first ones will receive special incentives and will be asked to provide useful best practices to others. 
Furthermore, the application will provide information regarding the results of the farmers who received help.
\\\\
The product needs to provide farmers the ability to visualize data relevant to them based on their location and type of production. 
Farmers should be also able to insert in the system data about production and any problem they face. 
They should be allowed to request help suggestions by agronomists or other farmers and create forums to discuss with their peers.
\\\\
The application will allow agronomists to insert the area they are responsible for, receive information about requests for help, 
and answer them. Agronomists need to know data about weather forecasts and the best performing farmers in the area; 
they also need to visualize and update daily plan visits of farms.

\subsection{Scope}
To represent the scope of the project we use the "The World and The Machine" model by M. Jackson. 
It contains the events which cannot be observed by the system ("The World"), 
those strictly related to the system ("The Machine"), and those in common between them. 

\subsubsection{World phenomena}
\begin{itemize}
    \item Meteorological event
    \item Policymaker implements policy
    \item Policymaker reward farmers
    \item Farmer's production
    \item Farmer use water to irrigate crops
    \item Agronomist checks farm
\end{itemize}

\subsubsection{Machine phenomena}
\begin{itemize}
    \item Policymaker data storage
    \item Farmer discussion forum
    \item Farmer performance evaluation
    \item Agronomist data storage
\end{itemize}

\subsubsection{Shared phenomena}
\paragraph{Controlled by the World}
\begin{itemize}
    \item Policymaker and agronomist check the farmers production rank
    \item Policymaker sign-in and log-in
    \item Farmer answers help request on the forum
    \item Farmer sign-in and log-in 
    \item Farmer add information about his production
    \item Farmer add information about a problem he faces
    \item Farmer create a discussion on the forum about a problem
    \item Farmer checks weather forecasts
    \item Farmer asks an agronomist for help
    \item Agronomist sign-in and log-in
    \item Agronomist checks help requests
    \item Agronomist answers help request on the forum
    \item Agronomist answers privately help request
    \item Agronomist checks weather forecasts
    \item Agronomist visualizes daily plan
    \item Agronomist confirms or modifies daily plan
\end{itemize}

\paragraph{Controlled by the Machine}
\begin{itemize}
    \item System sends personalized suggestions to the farmer
\end{itemize}

\subsubsection{Goals}

\begin{description}
    \item [G1] Policymakers shall be able to know if steering initiative produced significant results
    \item [G2] Policymakers shall be able to know the best and the worst farmer
    \item [G3] Farmers shall be able to communicate with peers
    \item [G4] Farmers shall be able to send an help request to an agronomist
    \item [G5] Farmers shall received personalized suggestions
    \item [G6] Farmers shall be able to check weather forecasts
    \item [G7] Farmers shall be able to add information about production and problems
    \item [G8] Farmers shall be able to create discussion on the forum
    \item [G9] Agronomists shall be able to visualize, confirm and modify their daily plan
    \item [G10] Agronomists shall be able to help farmers with problems
    \item [G11] Agronomists shall be able to know the best and the worst farmer
\end{description}

\subsection{Definitions, Acronyms, Abbreviations}

\subsubsection{Definitions}

\subsubsection{Acronyms}

\subsubsection{Abbreviations}

\subsection{Revision history}

\subsection{Reference documents}
\begin{description}
    \item [WeBeeP channel] - Project Assignment
    \item [The World \& The Machine] - M. Jackson, P. Zave 
\end{description}

\subsection{Document structure}
