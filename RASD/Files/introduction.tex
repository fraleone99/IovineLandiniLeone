\subsection{Purpose}
The covid-19 pandemic has greatly highlighted the need of building a resilient food system in India, 
where agriculture plays a pivotal role. 
This need is increased by the problems due to climate change that will impact everything from productivity to 
livelihoods across food and farm systems and is predicted to result in a 4\%-26\% loss in net farm income towards the end of the century.
In addition, according to Harvard Business Review, food demand is expected to increase between 59\% to 98\% by 2050.\\
For this reason policy makers, citizens, agronomists, and farmers should share data and information to achieve better results 
\\\\
Telangana region is an extended and populous state of India, whose economy is mainly driven by agriculture. 
To address the described above problem, Telangana's government want to design, 
develop and demonstrate anticipatory governance models for food system using digital 
public goods and community-centric approaches to strengthen data-driven policy-making in the state.
\\\\
The application aims to enable the acquisition, communication and combination of data provided by Telangana policymakers, 
farmers and agronomists as:
\begin{itemize}
    \item meteorological forecasts
    \item farmers' production
    \item amount of used water
    \item soil humidity
    \item agronomists' report
\end{itemize}
\bigskip
The application will allow policymakers to identify farmers who are performing well and those who are performing badly. 
As the first ones will receive special incentives and will be asked to provide useful best practices to others. 
Furthermore, the product will provide information regarding the results of the farmers who received help.