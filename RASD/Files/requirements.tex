\subsection{External interface Requirements}
The \emph{DREAM} frontend is a web application that can be accessed from web browsers, both from mobile and
desktop devices. The following section will give a comprehensive description in terms of hardware, software
and communication interfaces.

\bigskip
\subsubsection{Common users interfaces}
\textbf{Login and Registration}\\
When first opening the application, all the users are presented with the login page. If not already registered,
it is provided a registration button. In this section they are asked to provide: first name, second name,
a \emph{username}, an \emph{e-mail address} and a \emph{password}. The interface also shows a small button to switch
to the login page, in case the user has already signed in on another device or the session has expired.

\bigskip
\begin{figure}[H]
    \centering
    \includegraphics[scale=0.40]{Images/registration.png}
    \caption{Registration interface}
\end{figure}

\newpage
\begin{figure}[H]
    \centering
    \includegraphics[scale=0.40]{Images/log-in.png}
    \caption{Log-in interface}
\end{figure}

\textbf{Farmer's home page} \\
Provides different options. The user can insert data concerning production,
checks weather forecasts, privately request help, and reach the forum.
\begin{figure}[H]
    \centering
    \includegraphics[scale=0.45]{Images/farmerHomePage.png}
    \caption{Farmer's home page}
\end{figure}

\textbf{Agronomist's home page} \\
Provides the possibility of selecting the area of responsibility, answering private help requests,
checking the weather, accessing the forum, managing the daily plan, and checking the dashboard
\begin{figure}[H]
    \centering
    \includegraphics[scale=0.40]{Images/agronomistHomePage.png}
    \caption{Agronomist's home page}
\end{figure}

\textbf{Policymaker's home page} \\
Provides a table containing values for all farmers in the system. Every field of this table can be ordered.
\newline There are also graphs that sum up data about production.
\begin{figure}[H]
    \centering
    \includegraphics[scale=0.40]{Images/policymakerHomePage.png}
    \caption{Policymaker's home page}
\end{figure}

\newpage
\textbf{Forum - home page} \\
The forum section is accessible to both the farmers and the agronomists. Agronomists have the possibility
of taking a look at the requests and answer to them. Farmers, as agronomists, can answer to the forum threads
but can also create a new discussion.
\begin{figure}[H]
    \centering
    \includegraphics[scale=0.40]{Images/agronomistForum.png}
    \caption{Agronomist's forum home page}
\end{figure}
\bigskip
\begin{figure}[H]
    \centering
    \includegraphics[scale=0.40]{Images/farmerForum.png}
    \caption{Farmer's forum home page}
\end{figure}

\newpage
\textbf{Daily plan - home page} \\
The daily plan home page presents a calendar in which appointments day are marked in red.
By clicking on a date is possible to manage it, by adding or removing an appointment.
\begin{figure}[H]
    \centering
    \includegraphics[scale=0.40]{Images/dailyplan.png}
    \caption{Daily plan management home page}
\end{figure}


\bigskip
\subsection{Functional Requirements}
\subsubsection{Goal-Requirement mapping}
\begin{description}
    \item [G1] Policymakers shall be able to know if steering initiative produced significant results
        \begin{description}
            \item[R1] The system shall keep track of the farmer who requested help and of the suggestion that they received
            \item[R2] The system shall keep track of historical data for each zone
            \item[R3] The system shall keep track of meteorological events
            \item[R4] Farmers shall be able to create a help request specifying  the type of problem
            \item[R5] The system shall be able to record data about production inserted by the farmer
            \item[R6] Agronomist shall be able to insert significant prodution thresholds for each zone   
        \end{description}
    \item [G2] Policymakers shall be able to know the best and the worst farmer
        \begin{description}
            \item[R2] The system shall keep track of historical data for each zone
            \item[R3] The system shall keep track of meteorological events
            \item[R5] The system shall be able to record data about production inserted by the farmer
            \item[R6] Agronomist shall be able to insert significant prodution thresholds for each zone   
        \end{description}
    \item [G3] Farmers shall be able to communicate with peers
        \begin{description}
            \item[R7] Farmers shall be able to create discussion threads on the forum
            \item[R8] Farmers shall be able to reply to discussion on the forum
            \item[R13] The system shall be able to manage a forum
        \end{description}
    \item [G4] Farmers shall be able to send an help request to an agronomist
        \begin{description}
            \item[R1] The system shall keep track of the farmer who requested help and of the suggestion that they received
            \item[R4] Farmers shall be able to create a help request specifying  the type of problem
            \item[R9] The system shall send a notification to the Agronomists when a farmer has requested help
        \end{description}
    \item [G5] Farmers shall received personalized suggestions
        \begin{description}
            \item[R1] The system shall keep track of the farmer who requested help and of the suggestion that they received
            \item[R2] The system shall keep track of historical data for each zone
            \item[R3] The system shall keep track of meteorological events
            \item[R5] The system shall be able to record data about production inserted by the farmer
            \item[R6] Agronomists shall be able to insert significant prodution thresholds for each zone
            \item[R10] After an agronomist has registered a best practice or a suggestion to a farmer, the system should infer the farmers with similar characteristics and send them the same suggestion 
        \end{description}
    \item [G6] Farmers shall be able to check weather forecasts
        \begin{description}
            \item[R3] The system shall keep track of meteorological events
            \item[R11] The systems shall be able to connect to an external service to retrieve meteorological forecasts
            \item[R12] The system shall give Farmer the possibility to see weather forecasts
        \end{description}
    \item [G7] Farmers shall be able to add information about production and problems
        \begin{description}
            \item[R4] Farmers shall be able to create a help request specifying  the type of problem
            \item[R5] The system shall be able to record data about production inserted by the farmer
        \end{description}
    \item [G8] Farmers shall be able to create discussion on the forum
        \begin{description}
            \item[R5] The system shall be able to record data about production inserted by the farmer
            \item[R13] The system shall be able to manage a forum
        \end{description}
    \item [G9] Agronomists shall be able to visualize, confirm and modify their daily plan
        \begin{description}
            \item[R14] The system shall be able to record data about visits of agronomists
            \item[R15] The system shall be able to manage a calendar
            \item[R16] The system shall organize Agronomists' calendar in such a way that they visit each farm at least twice a year
        \end{description}
    \item [G10] Agronomists shall be able to help farmers with problems
    \begin{description}
        \item[R4] Farmers shall be able to create a help request specifying  the type of problem
        \item[R9] The system shall send a notification to the Agronomists when a farmer has requested help
        \item[R10] After an agronomist has registered a best practice or a suggestion to a farmer, the system should infer the farmers with similar characteristics and send them the same suggestion 
    \end{description}
    \item [G11] Agronomists shall be able to know the best and the worst farmer
    \begin{description}
        \item[R2] The system shall keep track of historical data for each zone
        \item[R3] The system shall keep track of meteorological events
        \item[R5] The system shall be able to record data about production inserted by the farmer
        \item[R6] Agronomist shall be able to insert significant prodution thresholds for each zone   
    \end{description}
\end{description}

\bigskip
\subsubsection{Scenarios}
\paragraph{Scenario 1} Daily plan confirmation\\
Tom is an agronomist that has just finished his work shift, so he logs in to the app and clicks on the daily plan icon.
The daily appointment with farmers just completed, follows exactly the plan previously determined, so he presses the
confirmation button.

\bigskip
\paragraph{Scenario 2} Daily plan modification\\
Martha is an agronomist that is planning his appointments with farmers. She logs in to DREAM and, from the home page,
opens the daily plan section; here she decided to remove an appointment planned for the 10th of December selecting the
corresponding date and then the deletion icon that updates the daily plan. After that, Martha, from the calendar,
selects the 28th of December, and then she presses on the addition icon below: doing that another section is opened in
which she can insert information about the appointment to plan (date, time, and the farm that she wants to visit).
Confirming the operation, she will come back to the daily plan section and the daily plan is updated.

\bigskip
\paragraph{Scenario 3} Agronomist's area selection\\
Marvin is an agronomist that has just completed the registration. After the registration and the log in,
he decides to select his area of responsibility from the map icon on the home page. In this area, a political map of
Telangana is shown and Marvin selects his responsibility zone. No errors occur so Marvin can select the confirmation
button and his area is stored.

\bigskip
\paragraph{Scenario 4} Forum discussion creation\\
Anthony is a farmer and has problems with soil humidity in his fields, so he decides to post a question on the forum.
In order to do that, he logs in and clicks on the forum icon. In the forum section, he selects the add discussion icon:
another section is opened in which he writes his question and post it.

\bigskip
\paragraph{Scenario 5} Farmer insert data\\
Jack is a farmer who recently finished collecting his harvest. He opens the app, logs in to the system. On his home page, 
he selects the Insert Data icon, he now can insert the amount and the type of crop he has harvested and then confirm.

\bigskip
\paragraph{Scenario 6} Farmer Help Request\\
Marty is a farmer who is having some problems in the cultivation of a crop and because he can't wait for a response on the 
forum decides to request help from an agronomist. He logs in on the app, selects the request help icon, inserts a brief description 
of the problem and then confirms that he wants to ask for the help of an agronomist. Victor is one of the agronomists responsible 
for the zone of Marty. After Marty's request, he receives a message containing a description of the problem. Victor conveys that the 
problem of Marty is a serious one so he decides to schedule a visit to his farm through the app functionality.

\bigskip
\paragraph{Scenario 7} Policymaker checks performance of farmer who received help\\
Tom is an official of the Telangana Department of Agriculture, after the recent flood he wants to know how many farmers have requested 
help and if this help has been effective. He logs in on the website of DREAM and can see the dashboard with some aggregate data. 
To see which farmers requested help he can select the previous month from the choice box and check which user had requested help, 
then he can see if in the current month the users who received help have overcome the thresholds set by the agronomist. He then wants 
to know who are the best performing farmers so he orders the table to show the users who have the greatest number of meteorological 
events but overcome the threshold nevertheless.